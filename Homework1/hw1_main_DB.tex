\documentclass[]{article}

% include Packages
\usepackage{amsmath,amssymb}
\usepackage{amsthm}

% Formatting text ranges
\setlength{\textwidth}{450pt}
\setlength{\textheight}{685pt}
\setlength{\topmargin}{-40pt}
\setlength{\hoffset}{-15mm}

% Define custom commands
\newcommand{\ind}[2]{{#1}_{\mathrm{#2}}}
\newcommand{\trp}{^T}
\newcommand{\fx}{f(x)}
\newcommand{\fsigx}{f(\sigma x)}
\newcommand{\fxtrp}{f \trp (x)}
\newcommand{\xone}{x_1}
\newcommand{\xtwo}{x_2}
\newcommand{\dotx}{\dot x}
\newcommand{\dotxone}{\dotx_1}
\newcommand{\dotxtwo}{\dotx_2}
\newcommand{\Rn}{\mathbb{R}^n}
\newcommand{\Nset}{\mathbb{N}}
\newcommand{\forntoinf}{\overset{n \rightarrow \infty}{\longrightarrow}}
\newcommand{\jacf}{\dfrac{\partial }{\partial x} \fx}
\renewcommand{\brack}[1]{\left[ #1 \right]}
\newcommand{\parenth}[1]{\left( #1 \right)}
\newcommand{\jacfbrack}{\brack{\jacf}}
\newcommand{\matleq}{\preccurlyeq}
\newcommand{\norm}[1]{\parallel \! \! #1 \! \! \parallel}
\newcommand{\partialx}{\frac{\partial}{\partial x}}
\newcommand{\dsig}{\mathrm{d} \sigma}
\newcommand{\intsigma}{\int_{0}^{1} \partialx \fsigx x \dsig}
\newcommand{\Qx}{Q(x)}
\newcommand{\Vx}{V(x)}
\newcommand{\xtrp}{x \trp}
\newcommand{\gammaxsig}{\gamma(x,\sigma)}
\newcommand{\xn}{x_n}
\newcommand{\xnull}{x_0}
\newcommand{\xN}{x_N}
\newcommand{\xk}{x_k}
\newcommand{\uk}{u_k}
\newcommand{\xtrpn}{x_n \trp}
\newcommand{\fxn}{f(\xn)}
\newcommand{\dotVx}{\dot{V}(x)}
\newcommand{\dVdtx}{\frac{\mathrm{d} V(x)}{\mathrm{d} t}}
\newcommand{\partialVx}{\frac{\partial V(x)} {\partial x}}
\newcommand{\matricks}[4]{\begin{pmatrix}#1 & #2 \\ #3 & #4 \end{pmatrix}}
\newcommand{\twovector}[2]{\begin{pmatrix}{ #1 }\\{ #2 }\end{pmatrix}}
\newcommand{\Ad}{A_d}
\newcommand{\Bd}{B_d}
\newcommand{\xkplus}{x_{k+1}}
%\newcommand{\dotx}{\dot{x}}
\newcommand{\tk}{t_{k}}
\newcommand{\tkplus}{t_{k+1}}
\newcommand{\Aeul}{\ind{A}{eul}}
\newcommand{\Beul}{\ind{B}{eul}}

%opening
\title{Optimal Control WS20/21: Homework 1}
\author{Daniel Bergmann}

\begin{document}

\maketitle


\begin{enumerate}
	\item[\bf a)] Discretize cost functional:\\
	 	\begin{equation}
			 J \approx  \xN\trp Q \xN + \sum_{k = 0}^{N-1} \xk \trp Q \xk + \uk\trp R \uk
	 	\end{equation}
		
	\item[\bf b)] Matrix representation of discretized linear dynamics:\\
	
	We know
	\begin{equation}
		x(t) = e^{At} x(0) + \int_{0}^{t} e^{A(t-\tau)} B u(\tau) d\tau \label{eq:varofconst}
	\end{equation}
		Discretizing  with time step size $ h $ \[ \xk \overset{\mathrm{def}}{=} x(kh) \overset{\mathrm{def}}{=} x(t_k) \] and inserting it into  \eqref{eq:varofconst} yields therefore for the state $ \xkplus $ the expression
		
		\begin{align}
			\xkplus &= e^{Ah(k+1)} \xnull + \int_{0}^{(k+1)h} e^{A((k+1)h-\tau)} Bu(\tau) d\tau\\
					&= e^{Ah}\left[ e^{Akh}\xnull + \int_{0}^{kh} e^{A(kh-\tau)}Bu(\tau) d\tau\right] + \int_{kh}^{(k+1)h} e^{A(hT+h-\tau)}Bu(\tau) d\tau.
		\end{align}
		Where we assume that the steering signal $ u $ is constant between the discretization time samples $ \tk. $
		We simplify this expression  by substituting with $ v(\tau) = kh + h - \tau $ and obtain
		\begin{align}
			\xkplus &= e^{Ah}\xk - \left( \int_{v(kh)}^{v((k+1)h)} e^{Av} dv \right) B\uk\\
			&= e^{Ah}\xk - \left(\int_{h}^{0} e^{Av} dv B \right) \uk.\\
					&= \underbrace{e^{Ah}}_{=:\Ad}\xk + \underbrace{\left(\int_{0}^{h} e^{Av} dv B \right)}_{=:\Bd}   \uk. \label{eq:exactdiscr}
		\end{align}
		This verifies the given Matrix representation.
	\item[\bf c)] 
		Euler Approximation: \begin{equation}
			\dotx(\tk) \approx \frac{x(\tkplus) - x(\tk)}{h}. \label{eq:eulerapprox} 
		\end{equation}
		Rearranging \eqref{eq:eulerapprox} yields 
		\begin{align}
			x(\tkplus) - x(\tk) &\approx h \dotx(\tk)\\
			&= hAx(\tk) + hBu(\tk)\\
			\Longleftrightarrow \quad x(\tkplus) &\approx (I+Ah)x(\tk) + hBu(\tk)\\
			 \xkplus &\approx \underbrace{(I+Ah)}_{=: \Aeul} \xk + hB\uk.
		\end{align}
		Relation to \eqref{eq:exactdiscr}:\\
		By the definition of the matrix exponential, we have
		\begin{equation}
			\Ad = e^{Ah} = \sum_{k=0}^{\infty} \frac{A^kh^k}{k!}.
		\end{equation}
		 Neglecting all quadratic and higher terms yields with $ \Ad $ from {\bf b)} \[ \Ad \approx I+Ah = \Aeul  \] which is exactly the matrix obtained by the euler approximation.
		
\end{enumerate}

\end{document}
